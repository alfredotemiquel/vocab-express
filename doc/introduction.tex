So far, Linked Data principles and practices are being adopted by an increasing number of data providers, getting as a result a global data space on the Web containing thousands of LOD datasets \cite{Heath_Bizer_2011}. In order to reuse, review or query a dataset published on the Web of Data, it is important to know the structure of the data. One step forward for knowing in depth the structure of the data is to explore the vocabulary that the dataset is using, and how the dataset is using such vocabulary.

There are available works such as (1) \emph{LODStats}\footnote{\footnotesize \url{http://stats.lod2.eu/}} that provides the information related with the vocabulary of given dataset, and (2) \emph{make-void} \footnote{\footnotesize \url{https://github.com/cygri/make-void}} that computes statistics about RDF files. However, LODStats is thought for the whole set of LOD datasets registered in The Data Hub \footnote{\footnotesize \url{http://thedatahub.com}}, and it is based on declarative descriptions of those datasets, and \emph{make-void} is thought for RDF files but not for RDF datasets.

In this paper we present vocab-express\footnote{\footnotesize \url{http://vocab-express.nodester.com/}}, a simple tool for exploring the vocabulary of a given dataset. The tool provides all the related information to the vocabulary of the dataset (1) list of all classes, (2) list of all the properties, (3) the number of instances of each class, (4) the number instances of each property, (5) the language of labels and comments, of the vocabulary elements.


