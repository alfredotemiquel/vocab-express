So far, Linked Data principles and practices are being adopted by an increasing number of data providers, getting as result a global data space on the Web containing hundreds of LOD datasets \cite{Heath_Bizer_2011}. In this context it is important to promote the reuse and linkage of datasets, and to this end, it is necessary to know the structure of datasets. One step forward for knowing in depth the structure of a given dataset is to explore the vocabulary used in the dataset, and how the dataset is actually using such vocabulary.

There are available works such as (1) \emph{LODStats}\footnote{\footnotesize \url{http://stats.lod2.eu/}} that provides the information related to the vocabulary used in a dataset, and (2) \emph{make-void} \footnote{\footnotesize \url{https://github.com/cygri/make-void}} that computes statistics about RDF files. However, LODStats is thought for the whole set of LOD datasets registered in The Data Hub \footnote{\footnotesize \url{http://thedatahub.com}}, and it is based on declarative descriptions of those datasets; and \emph{make-void} is thought for RDF files but not for RDF datasets.

In this paper we present vocab-express\footnote{\footnotesize \url{http://vocab-express.nodester.com/}}, a simple tool for exploring the vocabulary used in a given dataset. The tool provides all the related information of the vocabulary: (1) the list of all classes, (2) the list of all the properties, (3) the number of instances of each class, (4) the number instances of each property, (5) the language of labels and comments, of the vocabulary elements.


